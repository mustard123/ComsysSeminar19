\chapter{Title of My Seminar Work}
\markboth{Title of My Seminar Work}{}
\chaptauthors{My Name}

\Kurzfassung{%
This is the abstract.
It fits pretty much on one page and is definitely not longer.}

\newpage

\minitoc %table of contents

\newpage

\section{This is My First Section}

You only apply changes to the folder with your respective talk number.
This means that if your talk has number \texttt{X} you place all your files, e.g., pictures, \textbf{exclusively} in folder \texttt{TalkX}.
The (main) text of your seminar work goes in \texttt{Seminar-Arbeit.tex} in that folder.
Please use file \texttt{Example.tex} as a basis.
Formatting, page settings, and the file \texttt{talk.tex} must not be changed.

Do not -- under no circumstances -- change the file \texttt{talk.tex}.
If it is impossible to avoid the use of further packages (or modify the preamble in any other way) you may apply these modifications to \texttt{TalkX/MyHeader.tex}.
However, in this case it is important to consult your advisor beforehand, as \LaTeX \ does not contain namespaces, which may result in conflicts between different packages.

\section{Report Structure}

Your seminar report is contained in a chapter (\texttt{$\backslash$chapter}), wherefore you may use commands
\texttt{$\backslash$section\{\}},
\texttt{$\backslash$subsection\{\}}, and
\texttt{$\backslash$subsubsection\{\}}
to structure it.

In general, breaks need to be separated by an empty line but not
\texttt{$\backslash$$\backslash$} or \texttt{$\backslash$newline}. 
Please do not use \texttt{$\backslash$newpage}, \texttt{$\backslash$clearpage} etc.

Enumerations with and without numbers can be generated by use of the following commands:
\begin{quote}
  \begin{verbatim}
  \begin{enumerate}
    \item ...
    \item ...
  \end{enumerate}

  \begin{itemize}
    \item ...
    \item ...
  \end{itemize}
  \end{verbatim}
\end{quote}

For descriptions, the following command is suited:
\begin{quote}
  \begin{verbatim}
  \begin{description}
    \item[Term] Description
    \item[Term] Description
  \end{description}
  \end{verbatim}
\end{quote}


\section{Pictures and Tables}

Please embed \textbf{all} pictures without suffix and save the respective picture as \texttt{.jpg} or \texttt{.pdf} in folder \texttt{TalkX}. 
To embed pictures the following command can be used:
\begin{quote}
  \begin{verbatim}
  \begin{figure}[ht]
    \begin{center}
    \includegraphics[scale=0.6]{TalkX/filename}
    \end{center}
    \caption{Caption}
    \label{label}
  \end{figure}
  \end{verbatim}
\end{quote}

\begin{figure}[ht]
	\begin{center}
  \includegraphics[scale=0.6]{Example/uzh_logo_d}
  \end{center}
  \caption{Caption}
  \label{fig:label}
\end{figure}

Do always use relative paths to embed pictures!
To scale pictures you can also use \texttt{[width=4cm]} or
\texttt{[width=0.6$\backslash$textwidth]} instead of \texttt{[scale=0.6]}.
All pictures to be included in the seminar work need to be generated with a resolution of at least 600dpi.

  \begin{table}[h]
    \caption{Caption}
    \label{tab:label}
  	\begin{center}
    \begin{tabular}{|c|c|c|c|} \hline
	          & A & B & C \\ \hline\hline
	        X & 1 & 2 & 3 \\ \hline
	        Y & 4 & 5 & 6 \\ \hline
	        Z & 7 & 8 & 9 \\ \hline
	  \end{tabular}
	  \end{center}
  \end{table}

Table \ref{tab:label} can be generated by the following command.
\begin{quote}
  \begin{verbatim}
  \begin{table}
    \caption{Caption}
    \label{tab:label}
    \begin{center}
    \begin{tabular}{|c|c|c|c|} \hline
	          & A & B & C \\ \hline\hline
	        X & 1 & 2 & 3 \\ \hline
	        Y & 4 & 5 & 6 \\ \hline
	        Z & 7 & 8 & 9 \\ \hline
    \end{tabular}
    \end{center}
  \end{table}
  \end{verbatim}
\end{quote}

Pictures and tables need to have a caption (\verb|\caption|) and be referenced from within the running text by use of \texttt{$\backslash$ref\{label\}}.
In general, \texttt{caption} has to appear below pictures, but above tables!


\section{Bibliography}

The bibliography is placed at the end of your chapter. 
\textbf{Do not use marks on your bib\-items} as the automatically generated marks [1],[2],... are used.
For each reference the informations authors, title, publisher, and release date must be stated in the following form:

\begin{verbatim}
\bibitem {label} N. Author: Title of the document; Type of document 
  (technical report, deliverable, Workshop/Conference Name ...), 
  (Location, Vol. X, No. Y), Month, Year, pages, URL (if available).

\bibitem {label} Website title; \url{Website URL}, Month, Year of last visit.
\end{verbatim}
If the reference uses an URL the latter must be given by \texttt{$\backslash$url\{http://...\}}.

In running text, bibitems are referenced by the use of \texttt{$\backslash$cite\{label\}}.
For all papers, pictures and other works references need to appear at the according position.

A detailed instruction to the correct use of references can be found in \emph{Guideline to Written Seminar Works} \cite{leitfaden}. 

\section{Compiling}

\LaTeX \ is included in all popular Linux distributions.
Under Linux, the document is compiled by executing \texttt{pdflatex talk.tex} in the main directory, 
which generates \texttt{talk.pdf}. 

For Windows, the \TeX \ implementation MiKTeX (\url{http://www.miktex.org/}) in combination with the \LaTeX \ tool TeXnicCenter (\url{http://www.toolscenter.org/}) is recommended. For Mac OS X, the \TeX \ implementation MacTeX (\url{http://tug.org/mactex/}) in combination with the \LaTeX \ tool TeXShop (\url{http://pages.uoregon.edu/koch/texshop/}) is recommended.

Problems, proposals, and questions regarding the generation of your document can be sent by email to your supervisor. To submit your seminar talk compress (\texttt{zip} oder \texttt{tar}) the directory \texttt{TalkX} and mail it to your supervisor.


\begin{thebibliography}{99}
\bibitem {leitfaden} Martin Waldburger, Patrick Poullie, Burkhard Stiller: \emph{Guideline for Seminar Reports}, Communication Systems Group, Department of Infromatics, University of Zurich, January 2013. \url{http://www.csg.uzh.ch/teaching/guideline-seminar-report-v05.pdf}.
\end{thebibliography}

